\documentclass{article}[12pt,a4paper,twoside,draft]
\usepackage{texshade}
\usepackage{textopo}

\author{Samuel Young}
\date{August 18, 2018}
\title{Summary of an experiment -- The Genome Sequence and Evolution of Baculoviruses}

\begin{document}
\maketitle

I. setting of the experiment
   where? How? 
   Before there were dinsaours, there were a humongous insect's that when compared to the insect of today varitey are on the similar scale as dinaours to birds.  We are trying to find a way of knowing what viruses could have infected these insects an classfying a common ansectors from evolution. The use of a timemachine would be great but physically impossible. So instead we would like to try and find a passible work around. One passible way is too look at the viral genes and then compare them all at a relative frequnency of sites at which it shows up on a viruses genome where there gene should be placed and compared that to the other four groups viruses. In this study they found the 30 distinct genes that are overwhelming similar viral genes to to three of the other groups. when look at for certain set core genes to go about to compare the  gene analysis for there phylogentic tree. This study has been done another previous time but instead the had used it to generate  some more three new full sequenced genomes and compared the newly aquired to the  the original thrity genes that the had found by one of the previvous studies.     
   Where do i want to go with this statment?
   When we look at a certain population of Baculoviruses, we are looking for  population control of insect, so when the virus that infect these insects populations such as mosqiutoe's    
   why?
   Is the classifification of bacleovirus with speificy genes 

II. what is the knowledge that I am seeking for summmarizing the experiment 

III



\end{document}



