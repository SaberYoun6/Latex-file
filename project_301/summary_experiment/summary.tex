\documentclass{article}[12pt,a4paper,twoside,draft]
\usepackage{texshade}
\usepackage{textopo}

\author{Samuel Young}
\date{August 18, 2018}
\title{Summary of an experiment -- The Genome Sequence and Evolution of Baculoviruses}

\begin{document}
\maketitle

   Before there were dinsaours, there were a humongous insect's that when compared to the insect of today varitey are on the similar scale to dinaours as birds.
   That were trying to find a way of knowing what viruses could have infected these insects an classfying a common ansectors from evolutionary equivalent.
   Here an impractial use of time and a time-machine.
   So instead we would like to try and find a passible work around. 
   One passible way is too look at the viral genes and then compare them all at a relative frequnency at that site at which it pops on a viruses genome.
   Where there gene should be placed and compared that to the other four grouping of the this distinctive virus.
   In this study, they found that the had already had the thirty distinct genes and where just introducing more genomes into the the tree.
   They already had the set of core genes to go about and compare to the other the newly aquire genomes in order for them to do analysis of there phylogentic tree. 
   For reason the scientistthey tried to do a one-on-one comparsion to the viral genes in which that make did not lead to a consistent look tree.
   The then  made several different types of trees, some in which had went by different groupings for the graphing of the trees'. \\
   In order for them to determine the had to have a base for the amount of changes that had occured between the first graph and the final graphs.
   The scienist made the determination that the consenus was better then individuals data.
   The data made it seem that the individuals gene had there one evolutionary histories. 
   In which would result a distinct possibilty that all of the genes could be from different evolutions.
   Instead they used a majority consensus tree to determine the likelyness of these genes being in all four types of virus.
  In order for the data to graph into a tree the data must have a majority of the voting consensus  ( 50 \% ) with bootstrapping of higher measure of accuracy based on the predicted error of the thrity genes the were looking at.  
  The next tree that the printed looked at was a parsiminoius tree with the consensus of the 30 genes that the looked for the one with the fewest changes in the genomes form the grouping them together based sequences.
  All but one of the trees were rooted from the one with the farest consensus point this tree has a parsiminoious with the same common genes as discussed earlier. 
  The next tree looked at a threshold based on the on how far apart each genomes set of thrity genes had between the  distance from the other genomes.
   This tree was generated based solely on gene content as it was related to phyogenetic analysis.
   \\



\end{document}



