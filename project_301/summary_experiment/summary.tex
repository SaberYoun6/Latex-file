\documentclass{article}[12pt,a4paper,twoside,draft]
\usepackage{texshade}
\usepackage{textopo}

\author{Samuel Young}
\date{August 18, 2018}
\title{Summary of an experiment -- The Genome Sequence and Evolution of Baculoviruses}

\begin{document}
\maketitle

I. setting of the experiment
   where? How?\\ 
   Before there were dinsaours, there were a humongous insect's that when compared to the insect of today varitey are on the similar scale to dinaours as birds.
   They were trying to find a way of knowing what viruses could have infected these insects an classfying a common ansectors from evolutionary equivalent.
   Here an impractial use of time and a time-machine.
   So instead we would like to try and find a passible work around. 
   One passible way is too look at the viral genes and then compare them all at a relative frequnency of theres at that site at which it pops up at a viruses genome.
   Where there gene should be placed and compared that to the other four grouping of the this distinctive virus.
   In this study, they found that the had already had the thirty distinct genes and where just introducing more genomes into the the tree.
   They already had the set of core genes to go about and compare to the other the newly aquire genomes in order for them to do analysis of there phylogentic tree.\\ 
   For some of them they tried to do a one-on-one comparsion to the viral genes in which that make did not lead to a consistent look tree.
   The made several different types of trees, some in which had different functions in order for them to see how many distincitives changes had occured, between the first graph and the final graphs organization. \\
   Where do i want to go with this statment?
   When we look at a certain population of Baculoviruses, we are looking for  population control of insect, so when the virus that infect these insects populations such as mosqiutoe's    
   why?
   Is the classifification of bacleovirus with speificy genes 

II. what is the knowledge that I am seeking for summmarizing the experiment 

III



\end{document}



