\documentclass{article}
\usepackage{texshade}
\title{}
\author{Samuel Young}
\date{August, 29, 2018}

\begin{document}
 
 \maketitle

\title{Introduction of problem}\\
  Is when this scientist went looking for a phylogenetic tree of cyanophages are virus that infect blue-green bacteria also know as cyanobacteria, they didn't fine one. There investiagation then came across several individual articles that leaded to being type with together phages that also affect athropods. That wasn't what they were looking for instead they were looking wheither these two distinctive of cyanophages  types could have a similar gene. \\
     
\title{Decription of the experiment}\\
  First, the scientist wrote a code the that would pick ramdomly pick the genome "Synechococcus phage ACG-2014a". In the same script they also include within it was one to randomly pick out a gene number "68". Then they created at two seperate database with the use of BLAST where there was the single gene and they other database with all the genomes and genes. The sceinetist was thus looking for a otrhology .The first BLASTp that they then ran  as a database gene compared to all the other genomes . The second BLASTp that the ran against all the genomes as the database compared a majority of the other the genes. They went through both BLASTed files and found all the lowest e-values, in which they then compared the similar sizes of BLAST from both side. Considering that none of the base data pulled exactly the same. It was then decided to use the distance between the first and the second to determine how similar the were. Well in one file he had only most similar with that gene and in the other they had all the genes. Then from there they were aligned with all the genes that they had found with they uses of custal omega with the use of input method.\\
\title{Result of the experiment}\\
\begin{left}
\begin{tabular}{||c | c | c | c ||}
\hline
genomes & amino acids & matches & identical-matches \\ [0.5ex]
\hline\hline

\title{conclusion}\\
  Well instead of not knowing how many genomes I had I could have of just went with the amount of genes that are within this file. I could have of randomized that to determine that what the single one was looking for. It was only that that the  data that it put out with such similarites were 198.  Maybe next time I could try and find a better gene to match them all. Another thing that could  have of been done would to be to compare one speices form both genomes just to see the result of what one was closer for all the test. Another thing would be for me to determine which protein belong to what organism before i ran the file.\\

\end{document}

